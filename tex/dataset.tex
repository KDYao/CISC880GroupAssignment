\section{Dataset Description}
In this section, we describe how we collect the dataset that we used to answer our research questions.

\subsection{Data Preprocessing}

\begin{table}[]
	\centering
	\small
	\caption{Stack Overflow’s reputation system.}
	\label{tab:stackoverflow_rep}
\begin{tabular}{ll}
Action                & Reputation change    \\ \hline
Answer is upvoted     & +10                  \\
Answer is downvoted   & -2 (-1 to voter)     \\
Answer is accepted    & +15 (+2 to acceptor) \\
Question is upvoted   & +5                   \\
Question is downvoted & -2 (-1 to voter)     \\
Answer wins bounty    & +bounty amount       \\
Offer bounty          & -bounty amount       \\
Answer marked as spam & -100                
\end{tabular}
\end{table}


\begin{table*}[]
	\centering
	\small
	\caption{Statistics of the Stack Overflow dataset.}
	\label{tab:stackoverflow_stat}
\begin{tabular}{c|c|c}
Data      & Status (2008 to 2010)                  & Status (2015 to 2017)                   \\
Users     & 440K (198K questioners, 71K answerers) & 8.2M (1.9M questioners, 1.2M answerers) \\
Questions & 1M (69\% with accepted answer)         & 15M (55\% with accepted answer)         \\
Answers   & 2.8M (26\% marked as accepted)         & 24M (35\% marked as accepted)           \\
Votes     & 7.6M (93\% positive)                   & 100M (89\% positive)                    \\
Favorites & 775K actions on 318K questions         & 9.3M actions on 963K questions         
\end{tabular}
\end{table*}

There are 8 types of posts by users in Stack Overflow \cite{b1}. In this study, we are interested in the “Question” and “Answer” posts that are the primary motivation of the platform which drives the community activity. Any user with a registered account can post a “Question” in the site. Following the posting of the “Question” other users in the platform can post their “Answer”. There are further activates and actions that can be performed on a “Question” and “Answer”. 
There are 38 different types of actions in Stack Overflow \cite{b1}. For our study we are interested in the “Comment” and “Vote”. Comments are posted under a post and has a score given by users. There are 16 types of votes in Stack Overflow \cite{b1}. We are interested in 4 types of votes for our study, “AcceptedByOriginator”, “UpMod” AKA upvote, “DownMod” AKA downvote and “BountyStart”.  “AcceptedByOriginator” means that an answer to a question has been accepted as the correct answer to that question by the questioner. Upvote indicates how many distinct users found a question or answer to be useful. Downvote indicates how many distinct users found a question or answer to be not useful. The difference between the upvote and downvote on a post is considered the score of that post. “BountyStart” is a special vote put on a question only. Providing an answer to a question after a bounty has been placed on it gives the answerer an opportunity to receive more “Reputation” from it if his/her answer is accepted by the questioner.
User activity in the Stack Overflow is indicated by their “Reputation”. Users receive positive reputation points for upvotes in their posts and negative reputation points for downvotes. A detailed chart of reputation change is shown in Table~\ref{tab:stackoverflow_rep}. The reputation system in Stack Overflow is modelled to motivate users to post rich content and be active in the site. Different actions in the site is privileged to higher reputation users. Users require a minimum of 15 reputation points to vote on posts. A minimum of 75 points to put a bounty on a question. The amount of bounty offered is deducted from the user who put the bounty. The lowest bounty that can be offered is 50 and maximum is 500. Questioner decides who gets the bounty by accepting an answer at any point in time after the bounty was set.
We collected data about all question and answers between August 31, 2008 and December 31, 2010. The dataset consisted details of 299,398 distinct users, 1,096,144 questions, 2,632,009 answers, 4,657,336 comments and 10,143,364 votes. The dataset was archived in a xml format, from there we have extracted the required information and populated a SQL database. More details about the data is shown in Table~\ref{tab:stackoverflow_stat}.
