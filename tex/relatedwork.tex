\section{Related Work}
Researchers have been studying community Q\&A (question and answer) forums for a quite a long time from a different point of views. Few of the research groups focused on studying the users, their characteristics in community Q\&A forums as well as their motivations behind contribution \cite{adamic2008knowledge}, \cite{preece2004top}, \cite{nam2009questions}. The output of these studies has helped to develop the network-based ranking algorithms which identify users who have higher expertise \cite{jurczyk2007discovering}, \cite{nam2009questions}, \cite{yang2008crowdsourcing}, \cite{zhang2007expertise}.

Another group considered these Q\&A forums simply as the data source of questions and answers. They retrieve this information and treated the questions as query and the answers as the results \cite{liu2011predicting}, \cite{harper2008predictors}, \cite{liu2008predicting}, \cite{agichtein2009modeling}, \cite{jeon2006framework}. The aim of these kinds of studies is to find out a question according to the search query and propose the best answer relevant to that query. This approach can be considered as a trial to remove unnecessary information from the searched question pages and focusing on the best answers. Sometimes this kind of problems is treated as a classification problem which tries to predict whether the quality of the provided answer is high \cite{shah2010evaluating} in a given question. However, in our case, we found out that users are benefited from good answers of users with different level of expertise (according to their reputation). For each question, the answers build a thread of discussion which provides competing approaches. If any of these answers are read in isolation, it will lose its value. Models of inquiry noting networks as zero-sum two-sided markets of inquiry askers and answers have also risen \cite{kumar2010evolution} with the objective of clarifying the elements and steadiness of Q\&A communities.

Our work on predicting long-term value of the questions and the difficulty level of the questions are on the side of novelty and popularity of contents online \cite{wu2007novelty}, \cite{ratkiewicz2010characterizing}, \cite{szabo2010predicting}, which can also be considered as a part of the role of search engines in discovering online contents in a broader sense. Another line of research focused on discussion, voting and the feedback of the users in community Q\&A forums\cite{danescu2009opinions}, \cite{leskovec2010governance}, \cite{leskovec2010predicting}. Although these researches mainly aimed at finding out the behavior of users on voting, our focus is more on the identifying the quality of questions and answers with early community-based indicators. 

Finally, researchers have studied the Stack Overflow and similar kinds of Q\&A community forums like Stack Exchange to observe the relationship between user reputation and the quality of received answer \cite{tausczik2011predicting} in the past. Oktay et al. have tried to reveal the use of different quasi-experimental designs to build causal relationships in social networking sites by studying the dynamic arrival \cite{oktay2010causal} character of Stack Overflow answers. The observation from this research suggests that even if the best answer is accepted by the question owner, answers keep arriving which is somewhat related to our work. From our study, we found that these efforts by the user community can provide information which may not be necessary for the question asker to meet her current need.