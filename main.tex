\documentclass[conference]{IEEEtran}
\IEEEoverridecommandlockouts
% The preceding line is only needed to identify funding in the first footnote. If that is unneeded, please comment it out.
\usepackage{cite}
\usepackage{amsmath,amssymb,amsfonts}
\usepackage{algorithmic}
\usepackage{graphicx}
\usepackage{textcomp}
\usepackage{xcolor}
\usepackage{booktabs} % To thicken table lines
\usepackage{comment}
\def\BibTeX{{\rm B\kern-.05em{\sc i\kern-.025em b}\kern-.08em
    T\kern-.1667em\lower.7ex\hbox{E}\kern-.125emX}}
\begin{document}

\title{Paper Replication: Discovering Value from Community Activity on Focused Question Answering Sites: A Case Study of Stack Overflow\\
%\thanks{Final project report for CISC 880.}
\thanks{Group assignment for CISC 880.}
}

\author{\IEEEauthorblockN{Aaditya Bhatia, Abdullah Ahmad Zarir, Daniel Lee, Kundi Yao, Md Hasan Ibrahim}
\IEEEauthorblockA{\textit{Software Analysis and Intelligence Lab (SAIL)} \\
\textit{Queen's University}\\
Kingston, Canada \\
\{aaditya.bhatia, a.zarir, 18dil, 18ky10, ibrahim.mdhasan\}@queensu.ca}
%\and
%\IEEEauthorblockN{2\textsuperscript{nd} Given Name Surname}
%\IEEEauthorblockA{\textit{dept. name of organization (of Aff.)} \\
%\textit{name of organization (of Aff.)}\\
%City, Country \\
%email address}
}

\maketitle

\begin{abstract}

Question-answering (Q\&A) websites offer a plethora of meaningful knowledge that remains untapped. Prior studies mainly focused on providing the best answer to the questioner. However, there is a shift towards extrapolating value for a broader audience by analyzing community dynamics. The voting and reputation feature of Q\&A sites ensure the accuracy and quality of the content.


In our paper, we focused on questions with their corresponding set of answer, rather than individual question-answer pairs. In addition, we used reputation to find information about levels of community involvement and incentives for successful contributions and positive behavior. We considered the dynamics of the community activity from a question and full site level that shaped the set of answers, how answers and votes arrive over time and how the dynamics influence the outcome. We formulated two prediction tasks to help us predict the long-lasting value of a question and predict whether a question has been sufficiently answered. Based on the high performance of our prediction tasks, we concluded that there are certain features in community dynamics on Stack Overflow that can provide valuable information.

\input{tex/abstract}
\end{abstract}


\hfill \break
\noindent\textbf{Categories and Subject Descriptors:} H.3.4 [\textbf{Information Storage and Retrieval}]: Systems and Software.


\noindent\textbf{General Terms:} Experimentation, Human Factors.


\noindent\textbf{Keywords:} Question-answering, reputation, value prediction.

%\begin{IEEEkeywords}
%component, formatting, style, styling, insert
%\end{IEEEkeywords}

\section{Introduction} 

Question-answering (QA) websites are an on-line knowledge base where people seek solutions to their specific questions. 

\section{Related Work}
Researchers have been studying community Q\&A (question and answer) forums for a quite a long time from a different point of views. Few of the research groups focused on studying the users, their characteristics in community Q\&A forums as well as their motivations behind contribution \cite{adamic2008knowledge}, \cite{preece2004top}, \cite{nam2009questions}. The output of these studies has helped to develop the network-based ranking algorithms which identify users who have higher expertise \cite{jurczyk2007discovering}, \cite{nam2009questions}, \cite{yang2008crowdsourcing}, \cite{zhang2007expertise}.

Another group considered these Q\&A forums simply as the data source of questions and answers. They retrieve this information and treated the questions as query and the answers as the results \cite{liu2011predicting}, \cite{harper2008predictors}, \cite{liu2008predicting}, \cite{agichtein2009modeling}, \cite{jeon2006framework}. The aim of these kinds of studies is to find out a question according to the search query and propose the best answer relevant to that query. This approach can be considered as a trial to remove unnecessary information from the searched question pages and focusing on the best answers. Sometimes this kind of problems is treated as a classification problem which tries to predict whether the quality of the provided answer is high \cite{shah2010evaluating} in a given question. However, in our case, we found out that users are benefited from good answers of users with different level of expertise (according to their reputation). For each question, the answers build a thread of discussion which provides competing approaches. If any of these answers are read in isolation, it will lose its value. Models of inquiry noting networks as zero-sum two-sided markets of inquiry askers and answers have also risen \cite{kumar2010evolution} with the objective of clarifying the elements and steadiness of Q\&A communities.

Our work on predicting long-term value of the questions and the difficulty level of the questions are on the side of novelty and popularity of contents online \cite{wu2007novelty}, \cite{ratkiewicz2010characterizing}, \cite{szabo2010predicting}, which can also be considered as a part of the role of search engines in discovering online contents in a broader sense. Another line of research focused on discussion, voting and the feedback of the users in community Q\&A forums\cite{danescu2009opinions}, \cite{leskovec2010governance}, \cite{leskovec2010predicting}. Although these researches mainly aimed at finding out the behavior of users on voting, our focus is more on the identifying the quality of questions and answers with early community-based indicators. 

Finally, researchers have studied the Stack Overflow and similar kinds of Q\&A community forums like Stack Exchange to observe the relationship between user reputation and the quality of received answer \cite{tausczik2011predicting} in the past. Oktay et al. have tried to reveal the use of different quasi-experimental designs to build causal relationships in social networking sites by studying the dynamic arrival \cite{oktay2010causal} character of Stack Overflow answers. The observation from this research suggests that even if the best answer is accepted by the question owner, answers keep arriving which is somewhat related to our work. From our study, we found that these efforts by the user community can provide information which may not be necessary for the question asker to meet her current need.

\section{Dataset Description}
In this section, we describe how we collect the dataset that we used to answer our research questions.

\subsection{Data Preprocessing}

\begin{table}[]
	\centering
	\small
	\caption{Stack Overflow’s reputation system.}
	\label{tab:stackoverflow_rep}
\begin{tabular}{ll}
Action                & Reputation change    \\ \hline
Answer is upvoted     & +10                  \\
Answer is downvoted   & -2 (-1 to voter)     \\
Answer is accepted    & +15 (+2 to acceptor) \\
Question is upvoted   & +5                   \\
Question is downvoted & -2 (-1 to voter)     \\
Answer wins bounty    & +bounty amount       \\
Offer bounty          & -bounty amount       \\
Answer marked as spam & -100                
\end{tabular}
\end{table}


\begin{table*}[]
	\centering
	\small
	\caption{Statistics of the Stack Overflow dataset.}
	\label{tab:stackoverflow_stat}
\begin{tabular}{c|c|c}
Data      & Status (2008 to 2010)                  & Status (2015 to 2017)                   \\
Users     & 440K (198K questioners, 71K answerers) & 8.2M (1.9M questioners, 1.2M answerers) \\
Questions & 1M (69\% with accepted answer)         & 15M (55\% with accepted answer)         \\
Answers   & 2.8M (26\% marked as accepted)         & 24M (35\% marked as accepted)           \\
Votes     & 7.6M (93\% positive)                   & 100M (89\% positive)                    \\
Favorites & 775K actions on 318K questions         & 9.3M actions on 963K questions         
\end{tabular}
\end{table*}

There are 8 types of posts by users in Stack Overflow \cite{b1}. In this study, we are interested in the “Question” and “Answer” posts that are the primary motivation of the platform which drives the community activity. Any user with a registered account can post a “Question” in the site. Following the posting of the “Question” other users in the platform can post their “Answer”. There are further activates and actions that can be performed on a “Question” and “Answer”. 
There are 38 different types of actions in Stack Overflow \cite{b1}. For our study we are interested in the “Comment” and “Vote”. Comments are posted under a post and has a score given by users. There are 16 types of votes in Stack Overflow \cite{b1}. We are interested in 4 types of votes for our study, “AcceptedByOriginator”, “UpMod” AKA upvote, “DownMod” AKA downvote and “BountyStart”.  “AcceptedByOriginator” means that an answer to a question has been accepted as the correct answer to that question by the questioner. Upvote indicates how many distinct users found a question or answer to be useful. Downvote indicates how many distinct users found a question or answer to be not useful. The difference between the upvote and downvote on a post is considered the score of that post. “BountyStart” is a special vote put on a question only. Providing an answer to a question after a bounty has been placed on it gives the answerer an opportunity to receive more “Reputation” from it if his/her answer is accepted by the questioner.
User activity in the Stack Overflow is indicated by their “Reputation”. Users receive positive reputation points for upvotes in their posts and negative reputation points for downvotes. A detailed chart of reputation change is shown in Table~\ref{tab:stackoverflow_rep}. The reputation system in Stack Overflow is modelled to motivate users to post rich content and be active in the site. Different actions in the site is privileged to higher reputation users. Users require a minimum of 15 reputation points to vote on posts. A minimum of 75 points to put a bounty on a question. The amount of bounty offered is deducted from the user who put the bounty. The lowest bounty that can be offered is 50 and maximum is 500. Questioner decides who gets the bounty by accepting an answer at any point in time after the bounty was set.
We collected data about all question and answers between August 31, 2008 and December 31, 2010. The dataset consisted details of 299,398 distinct users, 1,096,144 questions, 2,632,009 answers, 4,657,336 comments and 10,143,364 votes. The dataset was archived in a xml format, from there we have extracted the required information and populated a SQL database. More details about the data is shown in Table~\ref{tab:stackoverflow_stat}.


\section{Description of tasks}
In this section, we aim to cater to the two fundamental questions prevalent in Q\&A sites, i.e. predicting the long-term value of the question and predicting whether a question has been adequately answered. This is done by comparing performance gains for different features used to predict the individual tasks.
\subsection{Predicting long-term value of a question}
Q\&A sites serve as an information warehouse providing long term information to multiple users and viewers. Questions with a long-lasting value attract a lot a community attention. The early signs displayed by a question are adequate to demonstrate its long-term effects. A question as a ‘fast’ phase when it will gather community attention and a ‘slow’ phase where it will present its long-term value to the community. With more page views, the value if a question is high, as more people refer to the question for information gain.
\subsection{Predicting whether a question has been sufficiently answered}
The second task focuses on those questions which have not been sufficiently answered and turn them into valuable resources. A questioner decides to offer bounties on a question, he feels is inadequately answered. On the other hand, questioner accepts an answer when he is convinced with the response. This phenomenon is further explored with predicting whether a question will be answered in task 2.

\section{Community dynamics of question answering}

\subsection{A reputation pyramid}

\subsection{The activity level of a question}
In the previous section, we illustrate the structure of users' reputation pyramid, which explains how answers arrival process relates to the user's reputation, as well as other phenomena from our observation. In this section, we would investigate more on voting, another significant mechanism of Stack Overflow. Except for merely answerers and questioners, other users can also involve in those QA sections by commenting, voting, etc.. The voting mechanism is not only applicable to answers but also to questions, and it is also considered as an important factor to reflect community involvement and evaluation. From our observation of Stack Overflow, we notice that questions with more answers are more likely to benefit from community involvement: answers will get higher votes and questions will receive more favorites. Highly active QA processes are more inclined to benefit from the community, instead of a competence among answerers themselves.  Based on our observation, our goal is to validate whether if the activity level is able to explain the feedback and evaluation from the community activities.

\textbf{Higher activity produces benefits.}
Like we discussed, unlike the features of general QA sites, Stack Overflow is programming-oriented, the questions are generally hard to be answered by majority community users. Furthermore, based on our observation from the previous section, answerers' arrival time is related to acceptance rate and answerers' reputation, it motivates answerers to answer a question quickly once it is posted. However, is there a competitive relationship existing among answerers?

%Fig6
%TODO
%2010
\begin{figure}[!t]
	\centering
	\includegraphics[width=0.8\columnwidth]{img/Fig6_2010.pdf}
	\caption{Fig 6 2010.}
	\label{fig:fig6_2010}
\end{figure}

\begin{figure}[!t]
	\centering
	\includegraphics[width=0.8\columnwidth]{img/Fig6_2017.pdf}
	\caption{Fig 6 2017.}
	\label{fig:fig6_2017}
\end{figure}

In order to answer the question above, we choose all the questions with exactly two answers as our target dataset. Suppose $r_{i}$ is the reputation of the \textit{i}-th answerer, and $v_{i}$ is the number of vote score of the \textit{i}-th answer. If there exists a trend where $v_{1}$ goes up while $v_{2}$ decreases, we consider there's a competitive relationship between two answerers. Now we set the value of $r_{i}$ unchanged as x axis, and compare the average vote score for both answerers. To begin with, we collected data from all questions with two answers, and separate answers according to answerers' reputation. Since the dataset is highly biased, majority users have very limited reputation, which means they are either non-active users or possibly non-questioner or answerer. Adopting data from this part of users can deviate our result, we set a threshold where lower reputation users of the two answerers should have reputation between 75 and 125. In order to make the graph easier to interpret, first we scale x axis, which represents higher reputation users' reputation, using a logarithmic(base 10). For each reputation score, we use an average value to present higher and lower answerers' reputation respectively. Instead of representing vote score for each reputation point, we smooth the curve by splitting the vote scores into groups, according to different reputations in x axis, and calculate the average value of the group. From Fig\ref{fig:fig6_2010} we notice that in most cases, answer vote scores from high and low reputation users have similar trend, they either decrease or increase together, but high reputation answerers' answer votes have higher variance. The corresponding pattern reveals that in most cases, the two answerers do not have a competitive relationship. Similarly, this trend can also be noticed during 2015 to 2017. However, we notice the later dataset have relatively less average vote score, this could probably indicate that although there are a great amount of newly introduced answers, the quality of such answers are not as good as before, or the question has been sufficiently answered, newly added answers does not contribute much to the question. 

\section{Prediction tasks}
In this section, we grab the phenomenon surrounding around community process of Stack Overflow question-answering. We demonstrate prediction of value of a question to the community and to the questioner in two subtasks: Predicting the long-term value of a question and predicting whether the has been sufficiently answered or not.
\subsection{Predicting long-lasting value}

\begin{figure}[!t]
    \centering
    \includegraphics[width=0.7\columnwidth]{img/task1.pdf}
    \caption{Results of pageview prediction. Notice strong absolute and also relative performance of our method. (left) Accuracy, (right) Area under ROC curve. Top row is for quartile division of page view, bottom row is for half division of page view.}
    \label{fig:task1}
\end{figure}


\begin{table*}[]
	\centering
	\small
	\caption{Features used for learning}
	\label{tab:pred}
\begin{tabular}{|l|l|}
\hline
Questioner features (SA)               & \begin{tabular}[c]{@{}l@{}}questioner reputation, \# of questioner’s questions and answers, questioner’s, \\ percentage of accepted answers on their previous questions.\end{tabular}                                                                                                                                                                                                                            \\ \hline
Activity and Q/A quality measures (SB) & \begin{tabular}[c]{@{}l@{}}\# of favorites, \# of page views, \# positive and negative votes on question, \\ \# of answers, maximum answerer reputation, highest answer score, \\ reputation of answerer who wrote highest scoring answer.\end{tabular}                                                                                                                                                          \\ \hline
Community process features (SC)        & \begin{tabular}[c]{@{}l@{}}answerer reputation, median answerer reputation, fraction of sum of answerer \\ reputations contributed by max answerer  reputation, sum of answerer reputations, \\ length of answer by highest-reputation answerer, \# of comments on answer by \\ highest-reputation answerer, length of highest-scoring answer, \\ \# of comments on highest-scoring answer.\end{tabular}         \\ \hline
Temporal process features (SD)         & \begin{tabular}[c]{@{}l@{}}Average time between answers, median time between answers, \\ minimum time between answers, time-rank of highest-scoring answer, \\ wall-clock time elapsed between question creation and highest-scoring answer, \\ time-rank of answer by highest reputation answerer, wall-clock time elapsed between \\ question creation and answer by highest-reputation answerer.\end{tabular} \\ \hline
\end{tabular}
\end{table*}


\begin{table}[]
	\centering
	\small
	\caption{Feature coefficients for prediction task 1}
	\label{tab:8features}
	\resizebox{\columnwidth}{!}{%
\begin{tabular}{l|c}
Feature                                                   & Coefficient \\ \hline
Number of answers +0.61                                   & + 0.09      \\
Sum of answer scores +0.47                                & + 4.72      \\
\# of questioner’s questions (log scale) -0.46            & - 0.21      \\
Length of highest-scoring answer +0.38                    & + 0.08      \\
Questioner’s reputation (log scale) +0.31                 & + 0.05      \\
Time for highest-scoring answer to arrive +0.22           & + 0.34      \\
\# comments on highest-scoring answer +0.19               & + 0.23      \\
\# comments on highest-reputation answerer’s answer +0.17 & + 0.07      \\ \hline
\end{tabular}}
\end{table}


The community value of a question can be reflected in the long-term impact of a question. While, multiple features like “favourite count” , “Page Views”, “Up Votes” and “Down Votes reflect the impact of the question, page view has chosen to be a balanced feature as it is the reflector of a question’s value. Unlike up votes, down votes and favourite count; page view has a significantly large value. However, the number of page views can pose to be noisy and need pre-processing before prediction tasks. 
The features that have been found out from the previous sections include: questioners’ factors, activity and Q/A factors, community factors, and temporal factors as shown in Table \ref{tab:pred}.


\begin{figure}[!t]
    \centering
    \includegraphics[width=\columnwidth]{img/Spearman.pdf}
    \caption{Hierarchical clustering of variables according to Spearman’s correlation}
    \label{fig:spearman}
\end{figure}


\textbf{Experiment Setup} To perform this prediction, we split the page view value into high and low. In the first case, this division is done by the mean, and data is split into half across the mean. In the second case, page view is split into 4 quadrants, taking the top most and the bottom most quartiles as high and low respectively. Quartile split provides reinforcement to the learnings and important features found in the case where page views are split by half. We reduced the questions with pageviews more than 5 million, to remove outliers and for generalizability of results.

Providing adequate time for a questions’ pageview value to mature itself is vital for the experiment. All the questions posted within the same month, 1 year before the last update of the stack-overflow data dump (31st December 2011) have been considered for the study, i.e. we considered questions with creation date between 1st December 2010 and 31st December 2010.  Most features required to predict the long-term value of a question arrive within a specified time of when question was posted. To further analyse this, we have predicted the model taking all features in 1, 3, 24 and 72 hour time-frames.

From the initial set of 27 features, we have performed correlation analysis, redundancy analysis and backward feature selection to obtain a core set of 8 features that sufficiently predict the long-term value of a question. In Fig \ref{fig:spearman}, we found that maximum reputation answer, sum of reputation, mean reputation were highly correlated to median reputation. Median reputation has been considered to reduce the total number of branches in the dendrogram. Redundancy analysis results concluded that the minimum time-gap between answers was a redundant feature. From the remaining set of 23 features, we performed downward feature selection to find a core set of 8 features as given in Table~\ref{tab:8features}. The coefficients in the figure reflect the effect of the features in predicting the high and low values of quartile page views for questions with features computed up to 3 days.

The results conclude that number of answers, sum of answer scores, number of questioner’s questions, length of highest scoring answer, time for highest scoring answer to arrive, number of comments on highest scoring answer, and comments on the highest reputation answerer’s answer impact the long-lasting value of a question. 
A comparison has been made with respect to baseline features. The model built with baseline features considers two features, sum of upvotes – sum of downvotes of a question, and the number of favourites of the question. 
In Fig~\ref{fig:task1} the AUC value increases less than 2 AUC units from one hour to three-hour time-frame. Similar is the case for accuracy value. The results demonstrate that most of the information required to predict the value of the question is obtained in 1 hour of the creation time of the question. 

\subsection{Predicting whether the question has been sufficiently answered}
Bounties are questioner’s way of expressing whether their questions have been sufficiently answered or not. Users with reputation lower than the minimum value required by Stack Exchange for posting bounties, do not experience the freedom to stress the value of their question.
For each question with bounty, we observe ‘k’ number of answers after which bounty was offered; and take a random sample of questions for which an answer was accepted after ‘k’ number of answers. To make the experiment fair, only the non-bounty questions, with questioner’s reputations lesser than the minimum required amount (75 reputation points) have been considered. 
The set of features Sa, Sb, Sc, Sd have been considered, taking Sa as the baseline. Since, the value of ‘k’ is constant for the experiment, answerers features remain constant per value of k. Addition of answerers features have been done in the original set of features Sa, Sb, Sc, Sd to update to Sa`, Sb`, Sc`, Sd` as shown in the table.

\begin{itemize}
	\item Sa` = Sa
	\item Sb` = Sb
	\item Sc` = Sa + (Sum of Upvotes + Sum of Down-Votes + Sum Answerers’ Reputation for each answer) 
	\item Sd` = Sd + (Time gap between answers)
\end{itemize}


\begin{table}[]
	\centering
	\small
	\caption{Number of questions after which bounty is offered}
	\label{tab:kvalues}
\begin{tabular}{|r|r|}
\hline
\multicolumn{1}{|c|}{k} & \multicolumn{1}{c|}{\#records} \\ \hline
1                       & 7810                           \\ \hline
2                       & 3208                           \\ \hline
3                       & 1406                           \\ \hline
4                       & 640                            \\ \hline
5                       & 276                            \\ \hline
6                       & 138                            \\ \hline
7                       & 110                            \\ \hline
8                       & 44                             \\ \hline
9                       & 34                             \\ \hline
10                      & 22                             \\ \hline
11                      & 12                             \\ \hline
12                      & 10                             \\ \hline
\end{tabular}
\end{table}


\begin{figure}[!t]
    \centering
    \includegraphics[width=0.7\columnwidth]{img/Fig10_left.pdf}
    \includegraphics[width=0.7\columnwidth]{img/Fig10_right.pdf}
    \caption{Each curve shows how, given the reputation of the second answerer on a two-answer question, the likelihood of answering second deviates from a uniform baseline as a function of the reputation of the first answerer. The curves on the left (showing the bottom three reputation levels) slope downward, indicating lower reputation levels are more likely to answer questions second if the first answerer also has low reputation; and the curves on the right (showing the top three reputation levels) slope upward, illustrating an analogous homophily by reputation effect}
    \label{fig:fig10}
\end{figure}



\begin{table}[]
	\centering
	\small
	\caption{Important features for predicting task 2}
	\label{tab:18features}
	\resizebox{\columnwidth}{!}{%
\begin{tabular}{|l|l|}
\hline
Sa` & \begin{tabular}[c]{@{}l@{}}questioner reputation, \#of questioner’s questions, and \# of \\ questioner’s answers\end{tabular}                                                                                                                                                     \\ \hline
Sb` & \begin{tabular}[c]{@{}l@{}}\# favorites on question, maximum answer score, maximum\\ answerer reputation, and positive and negative question votes\end{tabular}                                                                                                                   \\ \hline
Sc` & \begin{tabular}[c]{@{}l@{}}average answerer reputation, \# positive votes on last answer,\\ \# negative votes on 2nd answer, length of highest-scoring answer,\\ length of answer given by highest-reputation answerer, \\ and \# comments on highest-scoring answer\end{tabular} \\ \hline
Sd` & \begin{tabular}[c]{@{}l@{}}average time difference between answers, time difference between\\ last 2 answers, time-rank of highest-scoring answer, and time-rank\\  of answer, by highest-reputation answerer.\end{tabular}                                                       \\ \hline
\end{tabular}}
\end{table}



\textbf{Experimentation Results}
With different values of ‘k’, we show the results in table. It is also interesting to observe that as the value of k goes high, users rich in reputation provide bounties others’ questions. The result is shown in Table~\ref{tab:kvalues}.

Due to the absence of a baseline (as present in prediction task 1), we take agglomerative addition of features Sb`, Sc` and Sd` in a baseline Sa` feature set to compute AUC and Accuracy metrics as shown in Fig~\ref{fig:fig10}.

We found that a core set of 18 features that impact this prediction in Table~\ref{tab:18features}. It is noted that addition of set Sd` in Sa`b`c` increases both Accuracy and AUC by less than 2 units, indicating low impact of time-based features in predicting whether a question has been sufficiently answered. 




\section{Conclusion}
 The paper provides insight on Q\&A websites where questions not only provide immediate ailment to the questioner’s concerns but provide long term value information store for the community. The paper discusses community processes and their implications in question answering. We also find the features important for predicting the long-lasting value of a question, and the features that implicate whether a question has been adequately answered. As information content is increasingly growing in question-answering sites, the paper helps in providing insights in terms of answer content and the processes that produce them.


\bibliographystyle{IEEEtran}
\bibliography{ref}


\vspace{12pt}

\end{document}
